%%% Die folgende Zeile nicht ändern!
\section*{\ifthenelse{\equal{\sprache}{deutsch}}{Zusammenfassung}{Abstract}}
%%% Zusammenfassung:
Im Nachgang der Wahl von Donald Trump zum US-Präsidenten im Jahre 2016 machten einige Forscher und zahlreiche journalistische Medien die Öffentlichkeit auf eine neue Qualität eines bis dahin weniger bekannten Problems aufmerksam: Die massenhafte Einmischung von Trollen, also destruktiven Fake-Accounts, in politische Diskussionen in sozialen Netzwerken.\\
Das erklärte Ziel der vorliegenden Abschlussarbeit ist es zu untersuchen, ob die algorithmische Erkennung solcher Trolle auf Twitter zuverlässig möglich ist. Hierzu wurde in einem ersten Schritt ein geeigneter Datensatz präpariert. Dieser besteht zu einem Teil aus Tweets von einem aufgedeckten russischen Trollnetzwerk und zu einem anderen Teil aus Tweets von echten Twitter-Usern. In einem weiteren Schritt wurde ein Python-Programm entwickelt, welches die Anwendung fünf optimierter Klassifikationsverfahren aus dem Bereich des maschinellen Lernens auf den vorliegenden Datensatz realisiert. Die Grundmenge der Tweets wird bei jeder Ausführung in eine Trainingsmenge, welche den Verfahren als Lerneingabe dient und eine Testmenge, welche zur Beurteilung der Performanz herangezogen wird, aufgeteilt. Das Programm gibt für alle Objekte in der Testmenge auf Basis des Gelernten eine Vorhersage ab, ob ein Troll-Tweet oder ein Nichttroll-Tweet vorliegt.\\
Mithilfe des beschriebenen Programms wurden für jedes Klassifikationsverfahren Ergebnisse generiert, welche im Anschluss ausführlich evaluiert wurden. Hierbei ergab sich folgendes eindeutiges Bild: Der überwiegende Großteil der getesteten Tweets wurde von allen fünf Verfahren der jeweils richtigen Klasse Troll oder Nichttroll zugeordnet. Eine durchschnittliche Korrektklassifikationsrate von rund 91\%, die von zwei Verfahren deutlich übertroffen wird, zeugt davon, dass Verwechslungen nur äußerst selten auftreten. Die Forschungsfrage, ob die Erkennung von Trollen auf Twitter zuverlässig möglich ist, kann demnach definitiv bejaht werden.