\section{Abschließendes}\raggedbottom
\subsection{Fazit}
Zusammenfassend ergibt sich aus den Ausführungen dieser Arbeit folgende zentrale Erkenntnis: Die Trolle der Internet Research Agency, welche zwischen 2015 und 2018 auf Twitter aktiv waren, sind anhand ihrer schriftlich getätigten Äußerungen zuverlässig, also mit einer hohen Genauigkeit, algorithmisch erkennbar.\\
Um die Erkennung zu ermöglichen, wurde zu Beginn die Pipeline der Textklassifikation in all ihren Einzelheiten beleuchtet. Dazu wurde aufgezeigt, wie die vorliegenden Tweets in einzelne Worte zerlegt und anschließend als Merkmalvektoren repräsentiert werden können. Nachfolgend wurde die Reduktion der Vektordimensionen, welche aufgrund verbesserter Performance einen essentiellen Zwischenschritt darstellt, thematisiert. Als Kernelemente dieser Arbeit wurden fünf verschiedene Klassifikationsverfahren vorgestellt. Erläutert wurden neben der grundlegenden Funktionsweise auch vornehmbare Einstellungen, welche abhängig vom Datensatz die Qualität der Ergebnisse beeinflussen können. Vor der Anwendung der Verfahren wurde vorbereitend besprochen, mit welchen Gütemaßen die Qualität eines Klassifikationsergebnisses messbar ist.\\
Im Rahmen der praktischen Arbeit wurde für jedes Verfahren eine Hyperparameteroptimierung durchgeführt um herauszufinden, welche Parametereinstellungen zu einer bestmöglichen Klassifikationsgenauigkeit führen. Hierbei wurde das jeweilige Verfahren wiederholt mit verschiedenen Einstellungen auf den Datensatz mit IRA-Tweets und Tweets von echten Benutzern angewandt. Es ergab sich das Bild, dass alle Verfahren nach der Optimierung gute bis sehr gute Ergebnisse in allen beschriebenen Gütemaßen hervorbrachten. Das bedeutet, dass sowohl die IRA-Trolle als auch die echten Benutzer zuverlässig erkannt werden und dass Verwechslungen unwahrscheinlich sind.
\subsection{Ausblick}
Im Hinblick auf den beabsichtigten Beitrag zur Trollbekämpfung stellen sich an diesem Punkt weitere Fragen. Die Erkennbarkeit wurde in dieser Arbeit für die Angehörigen eines spezifischen russischen Trollnetzwerks belegt. Ungeklärt ist, inwiefern die Ergebnisse auf andere Gruppierungen übertragbar sind. Ein mögliches Merkmal, in dem solche Netzwerke voneinander abweichen können, ist die vertretene politische Ideologie. Hier könnte eine potentielle Forschungsfrage sein, ob Trolle unterschiedlicher Ausrichtungen (links, rechts oder andere) gleichermaßen erkennbar sind. Ein weiteres Unterscheidungsmerkmal von Interesse könnte die benutzte Sprache sein. In diesem Zusammenhang kann man beispielsweise untersuchen, ob auf Deutsch twitternde Trolle genauso gut von ihren Nichttroll-Pendants unterschieden werden können wie jene, die die englische Sprache benutzen. Voraussetzung für die Durchführung der vorgeschlagenen oder analogen Untersuchungen ist der Zugang zu oder das Erstellen von geeigneten Datensätzen.\\
Während die Erkennbarkeit von IRA-Trollen nachgewiesen wurde, so sind die Ursachen für diese Gegebenheit dennoch gänzlich unklar. Die Ergebnisse sind ein Indiz dafür, dass den Trollen im vorliegenden Datensatz bestimmte sprachliche Kennzeichen zu Eigen sind. Dabei kann es sich sowohl um Besonderheiten im Sprachstil (Syntax) als auch um Eigenheiten in der Themenwahl (Semantik) handeln. Hier kann für die Gewinnung von Erkenntnissen eine umfassende Datenanalyse mit verschiedensten Analysetechniken aus dem Bereich des Text Minings durchgeführt werden.\\
Weitere Untersuchungen könnten sich auch damit befassen, ob einzelne oder alle Klassifikationsverfahren noch weiter optimiert werden können. In dem fiktiven Fall der großflächigen Anwendung in einem sozialen Netzwerk wären bei Prüfung einer Großzahl von Accounts auch Fehlerquoten von rund 5\% (wie bei den Ergebnissen dieser Arbeit) für das Unternehmen nicht unbedingt akzeptabel. Aus diesem Grund wäre eine Maximierung der erreichten Klassifikationsgenauigkeit wünschenswert. Hierzu wäre es einerseits möglich, weitere Methoden der Merkmalsextraktion in Betracht zu ziehen. Berücksichtigt werden könnten an dieser Stelle beispielsweise Verfahren des \textit{Word Embeddings} wie Word2Vec, welche einen Text nicht syntaktisch, sondern semantisch repräsentieren. Die Ergebnisse der zuvor erwähnten Datenanalysen könnte hier womöglich die Wahl geeigneter Methoden vereinfachen. Weiterhin sollten zusätzliche Klassifikatoren in die Untersuchung mit einbezogen werden. Zur Anwendung kommen könnten zum Beispiel weitergehende Methoden des \textit{Deep Learnings} wie \textit{Recurrent Neural Networks} (RNN) oder \textit{Convolutional Neural Networks} (CNN).