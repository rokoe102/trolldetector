\section{Evaluation}\raggedbottom
Im nachfolgenden Kapitel sollen nun die Ergebnisse der praktischen Anwendung der Klassifikationstechniken auf die vorliegenden Datensätze ausgewertet werden. Nach der Besprechung jener Optimierungsergebnisse im Einzelnen, sollen die Perfomances der Verfahren abschließend noch in einen Gesamtzusammenhang gestellt und verglichen werden.
\subsection{Hyperparameteroptimierungen}
Zu Beginn soll jedes Verfahren einzeln auf die Eignung als Trollerkennungsinstrument untersucht werden. Hierzu werden die Ergebnisse der Hyperparameteroptimierungen diskutiert. Folgende Hyperparameter teilen sich dabei alle Verfahren, da die Merkmalsextraktion bei ihnen gleich abläuft:
\begin{table}[htb]
	\begin{center}
		\begin{tabular}{|c|c|}
			\hline
			Hyperparameter & gewählte Werte \\ \hline \hline
			Merkmalgewichtung & TF, TF-IDF \\ \hline
			Stoppwort-Filterung & keine, englisch\\ \hline
			n-Gramm-Extraktion & 1-Gramme, 1+2-Gramme\\ \hline			
		\end{tabular}
		\caption{Gemeinsame Hyperparameter aller Verfahren}\label{common-params}
	\end{center}
\end{table}
\subsubsection{k-Nearest-Neighbor-Algorithmus}
Beim KNN-Algorithmus sind neben den Hyperparametern, welche mit anderen Verfahren geteilt werden, die Abstandsmetrik und der k-Wert für die zu untersuchenden Nachbarn gegeben.\\
Tabelle \ref{results-knn} zeigt die mittleren Ergebnisse der Ausprägungen aller Hyperparameter einerseits im Vergleich untereinander und im Vergleich zur durchschnittlichen und zur besten Performance. Hieraus lassen sich verschiedene Schlussfolgerungen ziehen:
Insgesamt sind beim KNN-Algorithmus ausgezeichnete Zahlen zu beobachten. Im Maximum liegen  nahezu alle Kennzahlen über 95\%, positiv auffallend sind hier Werte der Segreganz und der Sensitivität von über 97\%. Durchschnittlich werden immer Punktzahlen von über 93\% erreicht. Ein Unterschied im Abschneiden unterschiedlicher Parametereinstellungen ist bei der Merkmalsgewichtung zu erkennen: TF erreicht hier ein wenig bessere Punktzahlen als TF-IDF. Die Filterung von englischen Stoppwörtern bringt im Mittel keine Verbesserung hervor. Extrahiert man nur 1-Gramme, anstatt 1-Gramme und 2-Gramme, erreicht man bis zu 1\% höhere Punktzahlen.\\
Bei den verfahrensspezifischen Hyperparametern gibt es nur wenige Schwankungen. Beim k-Wert deutet sich an, dass ein einstelliger Wert  besser abschneidet als ein zweistelliger. Die erzielten Verbesserungen belaufen sich jedoch nur auf höchstens 0,5\%. Bei den Metriken ist auffällig, dass unterschiedliche Arten der Abstandsmessungen nahezu keinen Unterschied in der Qualität hervorbringen, die Abweichungen betragen hier höchstens 0,1\%.\\
\begin{table}[htb]
	\begin{center}
		\begin{tabular}{|c|c|c|c|c|c|c|}
			\hline 
			Hyperparameter & Genauigkeit & Relevanz & Segreganz & Sensitivität & Spezifität & $F_1$ \\ \hline \hline
			TF & \textbf{0.951} & \textbf{0.937} & 0.965 & 0.964 & \textbf{0.937} & \textbf{0.949} \\ \hline
			TF-IDF  & 0.949 & 0.932 & \textbf{0.966} & 0.964 & 0.935 & 0.948 \\ \hline \hline
			engl. Stoppwörter  & 0.947 & 0.931 & 0.962 & 0.961 & 0.934 & 0.946 \\ \hline
			keine Filterung  & \textbf{0.953} & \textbf{0.937} & \textbf{0.968} & \textbf{0.967} & \textbf{0.940} & \textbf{0.952} \\ \hline \hline
			1-Gramme  & \textbf{0.955} & \textbf{0.941} & \textbf{0.969} & \textbf{0.967} & \textbf{0.967} & \textbf{0.954} \\ \hline 
			1+2-Gramme  & 0.945 & 0.928 & 0.962 & 0.961 & 0.961 & 0.944 \\ \hline \hline
			$k = 5$  & \textbf{0.953} & \textbf{0.938} & \textbf{0.967} & \textbf{0.966} & \textbf{0.940} & \textbf{0.952} \\ \hline 
			$k = 15$  & 0.950  & 0.934  & 0.965  & 0.964 & 0.936  & 0.949 \\ \hline 
			$k = 25$  & 0.948  & 0.931  & 0.964  & 0.962 & 0.934  & 0.947  \\ \hline \hline
			euklid. Metr.  & 0.950 & \textbf{0.935} & 0.965 & 0.963 & 0.937 & 0.949 \\ \hline
			Manhattan  & 0.950 & 0.934 & \textbf{0.966} &\textbf{0.964} & 0.937 & 0.949 \\ \hline
			\hline
			Maximum  & 0.960 & 0.948 & 0.973 & 0.972 & 0.950 & 0.959 \\ \hline
			durchschnittl. & 0.950 & 0.934 & 0.965 & 0.964 & 0.937 & 0.949 \\ \hline
		\end{tabular}
		\caption{Ergebnisse des kNN-Algorithmus}\label{results-knn}
	\end{center}
\end{table}\\
Angesichts seiner durchweg hohen Punktzahlen in allen möglichen Gütemaßen ist der k-Nearest-Neighbor-Algorithmus sehr gut für die Erkennung von IRA-ähnlichen Trollen geeignet.
\subsubsection{Naiver Bayes-Klassifikator}
Der einzige verfahrensspezifische Hyperparameter des Naiven Bayes-Klassifikators ist die angenommene Verteilung der Merkmalvektoren. Getestet wurde mit einer Normalverteilung, einer Multinomialverteilung und einer komplementären Verteilung nach \citet{rennie03}.
\begin{table}[htb]
	\begin{center}
		\begin{tabular}{|c|c|c|c|c|c|c|}
			\hline 
			Hyperparameter & Genauigkeit & Relevanz & Segreganz & Sensitivität & Spezifität & $F_1$ \\ \hline \hline
			TF         & \textbf{0.747} & 0.772 & \textbf{0.767} & \textbf{0.685} & 0.804 & \textbf{0.694} \\ \hline
			TF-IDF     & 0.702 & \textbf{0.798} & 0.736 & 0.571 & \textbf{0.824} & 0.536 \\ \hline \hline
			engl. Stoppwörter  & 0.715 & 0.751 & 0.731 & 0.620 & 0.803 & \textbf{0.616} \\ \hline
			keine Filterung    & \textbf{0.734} & \textbf{0.819} & \textbf{0.77}1 & \textbf{0.636} & \textbf{0.825} & 0.614 \\ \hline \hline
			1-Gramme    & 0.735 & 0.811 & 0.761 & 0.642 & 0.642 & 0.631 \\ \hline 
			1+2-Gramme  & 0.713 & 0.758 & 0.741 & 0.613 & 0.613 & 0.599 \\ \hline \hline
			Normal      & 0.807 & 0.762 & 0.868 & 0.878 & 0.741 & 0.815 \\ \hline 
			Multinomial & 0.585 & 0.841 & 0.568 & 0.198 & 0.947 & 0.253 \\ \hline 
			Komplementär& 0.781 & 0.752 & 0.818 & 0.808 & 0.755 & 0.777  \\ \hline 
			\hline
			Maximum        & 0.834 & 1.000 & 0.949 & 0.958 & 1.000 & 0.855 \\ \hline
			durchschnittl. & 0.724 & 0.785 & 0.751 & 0.628 & 0.814 & 0.615 \\ \hline
		\end{tabular}
		\caption{Ergebnisse der Naiven Bayes-Klassifikation}\label{results-nb}
	\end{center}
\end{table}\\
Die Ergebnisse in Tabelle \ref{results-nb} lassen folgende Schlüsse zu:
Dieses Verfahren schneidet je nach Parameter-Einstellung sehr unterschiedlich ab. So liefert die Annahme einer Normalverteilung oder einer komplementären Verteilung Punktzahlen von 75 - 85\%.
Bei einer Multinomialverteilung sind die Ergebnisse im Durchschnitt deutlich schlechter. Beispielsweise ist die mittlere Treffergenauigkeit 58,5\% und die mittlere Sensitivität bei nur 19,8\%. Es gibt hierbei sehr starke, gleichzeitige Ausreißer nach oben und unten: In Kombination mit TF-IDF wird eine Spezifität von 100\% bei einer Sensitivität von 0\% erreicht, weshalb die Treffergenauigkeit hier etwa 50\% beträgt.\\
Bei der Merkmalgewichtung schneiden TF und TF-IDF in einer Hälfte der Gütemaße besser ab. Das Unterlassung einer Filterung von englischen Stoppwörtern führt auch bei diesem Verfahren zu leicht besseren Ergebnissen von 2 - 6\%.\\
Lässt man die Annahme einer Multinomialverteilung außen vor, so werden mit dieser Methode Punktzahlen von durchschnittlich 75\% - 85\% erreicht, was gerade mit Einstellung der richtigen Hyperparameter zu relativ verlässlichen Ergebnissen führt. Unter diesen Voraussetzungen ist das Verfahren zur Trollerkennung durchaus geeignet.
\subsubsection{Support Vector Machine}
Die Klassifikation mit einer Support Vector Machine hat wie in Kapitel \ref{svm} beschrieben den Bestrafungsparameter C als einzigen verfahrensspezifischen Hyperparameter.
\begin{table}[htb]
	\begin{center}
		\begin{tabular}{|c|c|c|c|c|c|c|}
			\hline 
			Hyperparameter & Genauigkeit & Relevanz & Segreganz & Sensitivität & Spezifität & $F_1$ \\ \hline \hline
			TF         & \textbf{0.875} & 0.813 & \textbf{0.959} & \textbf{0.964} & 0.793 & 0.882 \\ \hline
			TF-IDF     & 0.868 & \textbf{0.838} & 0.900 & 0.900 & \textbf{0.837} & \textbf{0.868} \\ \hline \hline
			engl. Stoppwörter  & 0.865 & 0.816 & 0.929 & 0.931 & 0.804 & 0.869 \\ \hline
			keine Filterung    & \textbf{0.878} & \textbf{0.834} & \textbf{0.930} & \textbf{0.933} & \textbf{0.826} & \textbf{0.880} \\ \hline \hline
			1-Gramme   & 0.870 & 0.820 & \textbf{0.932} & \textbf{0.935} & \textbf{0.935} & 0.874 \\ \hline
			1+2-Gramme  & \textbf{0.873} & \textbf{0.830} & 0.928 & 0.929 & 0.929 & \textbf{0.876} \\ \hline \hline
			$C=1.00$ & 0.872 & 0.826 & 0.930 & 0.932 & 0.815 & 0.875 \\ \hline 
			$C=0.75$ & 0.871 & 0.825 & 0.929 & 0.932 & 0.815 & 0.875 \\ \hline 
			$C=0.50$ & 0.871 & 0.825 & 0.930 & 0.932 & 0.814 & 0.875 \\ \hline
			\hline
			Maximum        & 0.892 & 0.867 & 0.970 & 0.974 & 0869 & 0.892 \\ \hline
			durchschnittl. & 0.871 & 0.825 & 0.930 & 0.932 & 0.815 & 0.875 \\ \hline
		\end{tabular}
		\caption{Ergebnisse der SVM-Klassifikation}\label{results-svm}
	\end{center}
\end{table}\\
Bei Ausführung der Optimierung kommen folgende Ergebnisse (Tabelle \ref{results-svm}) zustande:\\
Insgesamt liefert die SVM im Mittel Punktzahlen von mindestens 80\% in allen betrachteten Gütemaßen. Beste Werte sind Segreganz und Sensitivität: Durchschnittlich werden hier ca. 93\% und im Maximum 97\% erreicht.\\
Unterschiedliche Merkmalgewichtungen führen auch hier zu unterschiedlichen Ergebnissen. Sowohl TF als auch TF-IDF schneiden in drei von sechs Gütemaßen besser ab als das jeweils andere. Der stärkste Unterschied betrifft Segreganz und Sensitivität: Hier schneidet TF 6\% besser ab. Bei allen anderen Arten der Merkmalsextraktion gibt es nur wenige Schwankungen von höchstens 2\%.\\
Beim Bestrafungsparameter C sind bis auf 0,1\% keine Schwankungen in den unterschiedlichen Ausprägungen auszumachen.\\
Insgesamt ist die SVM-Klassifikation mit den erreichten Punktzahlen ziemlich verlässlich und daher für die Trollerkennung gut geeignet. Von Vorteil ist auch, dass die durchweg hohe Segreganz für ein niedriges $f_n$ spricht, was eine konservative Klassifikation, welche für die Trollerkennung wichtig ist, gewährleistet. Eine positive Eigenschaft ist außerdem die Stabilität des Verfahrens: Selbst mit wenigen bewussten oder zufälligen Einstellungen liefert das Verfahren immer noch gute Ergebnisse, was für eine Art Benutzerfreundlichkeit spricht.
\subsubsection{Entscheidungsbäume}
Bei der Entscheidungsbaum-Klassifikation sind neben den von allen geteilten Hyperparametern auch das Aufteilungskriterium gegeben. Die zwei Möglichkeiten sind hier die in Kapitel \ref{tree} beschriebenen Maße Gini Impurity und Entropie. Aus Tabelle \ref{results-tree} lassen sich folgende Schlüsse ziehen:\\
Die Klassifikation erreicht im Mittel in allen Gütekriterien sehr hohe Werte von 94\%. Die maximalen Werte liegen mit ca. 95\% nur 1\% darüber. Es lässt sich aus diesem Grund leicht erkennen, dass hier nur sehr geringe Schwankungen vorhanden sind. Dies ist auch im Vergleich unter den Hyperparametern zu erkennen: Zwischen TF und TF-IDF, der Filterung englischer Stoppwörter und keiner Filterung und der Auswahl der jeweiligen n-Gramme liegen jeweils nur 1\% Unterschied. Die Aufteilungskriterien Gini Impurity und Entropie unterscheiden sich nur um 0,2\%.\\
Diese ausgezeichneten Punktzahlen legen nahe, dass eine zuverlässige Trollerkennung mit diesem Verfahren sehr gut zu leisten ist. Sowohl die geringen Schwankungen unter den Hyperparametern, als auch die geringen Unterschiede in den Gütekriterien sprechen für eine hohe Stabilität des Verfahrens. Dies bedeutet auch hier, dass selbst zufällige Einstellungen niemals zu schlechten Ergebnissen führen können.\\
\begin{table}[htb]
	\begin{center}
		\begin{tabular}{|c|c|c|c|c|c|c|}
			\hline 
			Hyperparameter & Genauigkeit & Relevanz & Segreganz & Sensitivität & Spezifität & $F_1$ \\ \hline \hline
			TF       & 0.945 & 0.938 & 0.951 & 0.948 & 0.941 & 0.943 \\ \hline
			TF-IDF   & 0.936 & 0.930 & 0.941 & 0.937 & 0.934 & 0.934 \\ \hline \hline
			engl. Stoppwörter & 0.938 & 0.932 & 0.943 & 0.939 & 0.936 & 0.936 \\ \hline
			keine Filterung    & 0.942 & 0.935 & 0.949 & 0.946 & 0.939 & 0.941 \\ \hline \hline
			1-Gramme   & 0.944 & 0.937 & 0.951 & 0.948 & 0.948 & 0.943 \\ \hline
			1+2-Gramme  & 0.936 & 0.931 & 0.941 & 0.937 & 0.937 & 0.934 \\ \hline \hline
			Gini & 0.939 & 0.933 & 0.945 & 0.942 & 0.937 & 0.937 \\ \hline
			Entropie & 0.941 & 0.935 & 0.947 & 0.944 & 0.939 & 0.939 \\ \hline
			\hline
			Maximum        & 0.950 & 0.944 & 0.957 & 0.954 & 0.947 & 0.949 \\ \hline
			durchschnittl. & 0.940 & 0.934 & 0.946 & 0.943 & 0.938 & 0.938 \\ \hline
		\end{tabular}
		\caption{Ergebnisse der Entscheidungsbaum-Klassifikation}\label{results-tree}
	\end{center}
\end{table}\\
\subsubsection{Mehrschichtiges Perzeptron}
Der wichtigste einstellbare Hyperparameter bei der Klassifikation mit einem Mehrschichtigen Perzeptron ist die Aktivierungsfunktion. Bei der Hyperparameteroptimierung wurden die drei in Kapitel \ref{mlp} beschriebenen Aktivierungsfunktionen getestet.
\begin{table}[htb]
	\begin{center}
		\begin{tabular}{|c|c|c|c|c|c|c|}
			\hline 
			Hyperparameter & Genauigkeit & Relevanz & Segreganz & Sensitivität & Spezifität & $F_1$ \\ \hline \hline
			TF       & \textbf{0.904} & 0.868 & \textbf{0.947} & \textbf{0.948} & 0.863 & \textbf{0.906} \\ \hline
			TF-IDF   & 0.888 & 0.868 & 0.909 & 0.907 & \textbf{0.870} & 0.887 \\ \hline \hline
			engl. Stoppwörter & 0.891 & 0.862 & 0.925 & 0.924 & 0.860 & 0.892 \\ \hline 
			keine Filterung   & \textbf{0.901} & \textbf{0.873} & \textbf{0.931} & \textbf{0.931} & \textbf{0.873} & \textbf{0.901} \\ \hline\hline
			1-Gramme   & \textbf{0.898} & 0.868 & \textbf{0.932} & \textbf{0.932} & \textbf{0.932} & \textbf{0.898} \\ \hline
			1+2-Gramme  & 0.894 & 0.868 & 0.925 & 0.923 & 0.923 & 0.894 \\ \hline \hline
			relu & \textbf{0.931} & \textbf{0.917} & \textbf{0.945} & \textbf{0.942} & \textbf{0.920} & \textbf{0.929} \\ \hline
			tanh & 0.887 & 0.858 & 0.918 & 0.918 & 0.858 & 0.887 \\ \hline
			logistic & 0.871 & 0.829 & 0.922 & 0.923 & 0.822 & 0.873 \\ \hline
			\hline
			Maximum        & 0.937 & 0.928 & 0.968 & 0.972 & 0.933 & 0.936 \\ \hline
			durchschnittl. & 0.896 & 0.868 & 0.928 & 0.928 & 0.866 & 0.896 \\ \hline
		\end{tabular}
		\caption{Ergebnisse der MLP-Klassifikation}\label{results-mlp}
	\end{center}
\end{table}\\
Tabelle \ref{results-mlp} sind nun folgende Aussagen zu entnehmen: Im Durchschnitt liefert die MLP-Klassifikation Werte von mindestens 86\%. Maximal werden Punktzahlen von 93 - 97\% erreicht. Letztere Werte sind überwiegend auf die Aktivierungsfunktion ReLu zurückzuführen, welche in allen Gütemaßen 3 - 6\% besser abschneidet als die anderen beiden.\\
Auch bei den Parametern der Merkmalsextraktion sind diesmal deutlichere Unterschiede festzustellen. Die Merkmalgewichtung mit TF liefert, analog zu allen anderen Klassifikationsverfahren, in den meisten Gütemaßen deutlich bessere Ergebnisse als TF-IDF. Filtert man englische Stoppwörter aus den Tweets, führt dies zu 1\% schlechteren Ergebnissen, als wenn man dies unterlassen würde. Die ausschließliche Extraktion von 1-Grammen ist leicht besser als die Extraktion von 1- und 2-Grammen.\\
Die MLP-Klassifikation ist durch ihre hohen Punktzahlen in all ihren Varianten grundsätzlich für die Trollerkennung geeignet. In besonderem Maße gilt dies, wenn ReLu als Aktivierungsfunktion eingesetzt wird. Mit gut eingestellten Hyperparametern ist dieses Verfahren äußerst verlässlich. Interessant ist in diesem Zusammenhang folgendes: Die Trainingsphase mit Backpropagation wird bei einem Mehrschichtigen Perzeptron dann beendet, wenn sich der Klassifikationsscore $n$ Iterationen in Folge nicht um einen Toleranzwert $tol$ ändert. Um eine akzeptable Laufzeit zu gewährleisten, wurden die Werte bei der Hyperparameteroptimierung auf $n = 5$ und $tol = 0,25\%$ gesetzt. Dies bedeutet, dass sich bei Inkaufnahme einer langen Laufzeit die Punktzahlen in dem ein oder anderen Gütemaß um wenige Prozentpunkte verändern können.
\pagebreak
\subsection{Verfahren im Vergleich}
Abschließend werden nun alle Ergebnisse in einen Gesamtzusammenhang eingeordnet. Tabelle \ref{results-all} zeigt die Ergebnisse einer Rastersuche über alle fünf optimierten Algorithmen der Textklassifikation.  
\begin{table}[htb]
	\begin{center}
		\begin{tabular}{|c|c|c|c|c|c|c|}
			\hline 
			Verfahren & Genauigkeit & Relevanz & Segreganz & Sensitivität & Spezifität & $F_1$ \\ \hline \hline
			KNN      & \textbf{0.958} & \textbf{0.943} & \textbf{0.973} & \textbf{0.972} & \textbf{0.945} & \textbf{0.957} \\ \hline
			NB   & 0.830 & 0.779 & 0.898 & 0.907 & 0.759 & 0.838 \\ \hline 
			SVM & 0.892 & 0.866 & 0.919 & 0.918 & 0.868 & 0.892 \\ \hline 
			Baum   & 0.943 & 0.937 & 0.950 & 0.947 & 0.940 & 0.942 \\ \hline
			MLP   & 0.934 & 0.926 & 0.943 & 0.939 & 0.929 & 0.932 \\ \hline \hline
			durchschnittl. & 0.911 & 0.890 & 0.937 & 0.937 & 0.888 & 0.912 \\ \hline
		\end{tabular}
		\caption{Ergebnisse im Vergleich}\label{results-all}
	\end{center}
\end{table}\\
Im Durchschnitt werden mit allen Verfahren sehr gute Punktzahlen von mindestens 89\% erreicht. Bei Segreganz und Sensitivität sind sogar fast 94\% zu beobachten. Diese Werte sprechen dafür, dass sowohl Troll-Tweets als auch Nichttroll-Tweets mit einer hohen Zuverlässigkeit erkannt werden und dass die Gefahr einer Verwechslung eher gering ist. Somit kann für alle Verfahren eine grundsätzliche Eignung als Trollerkennungsinstrument festgestellt werden. Welches der fünf vorliegenden Methoden in einem konkreten Fall zur Anwendung kommen sollte ist jedoch stets abhängig von den an die Ergebnisse gestellten Anforderungen.\\
Ist Genauigkeit und Zuverlässigkeit das maßgebliche Kriterium, so ist die Wahl des k-Nearest-Neighbor-Algorithmus zu empfehlen. Dieser hat mit 94 - 97\% erkennbar die besten Ergebnisse aller Verfahren. Die Gefahr der Falschklassifikation ist hier mit unter 5\% besonders gering.\\
Wird jedoch Wert auf eine gute Laufzeit gelegt, so ist der kNN-Algorithmus nicht die erste Wahl. Bei einer gegebenen Anzahl der Trainingsvektoren $N$ und die Anzahl der Dimensionen $d$ kann man die Zeitkomplexität mit $\Theta(dN + kN)$ angeben, sofern die Abstände gespeichert werden und mit $\Theta(kdN)$, falls dies nicht der Fall ist. Die Entscheidungsbaum-Klassifikation hat bei fast gleichwertiger Zuverlässigkeit eine Worst-Case-Laufzeit von $\mathcal{O}(d N \cdot \log(N))$ \citep{sani18}, schneidet jedoch im Mittel deutlich besser ab und hat hier deshalb einen entscheidenden Vorteil.\\
Es sind in der Anwendung auch Szenarien denkbar, in denen eine Flexibilität bezüglich des Tradeoffs Genauigkeit versus Laufzeit gewünscht wird. Für diesen Fall ist die Klassifikation mit einem Mehrschichtigen Perzeptron zu empfehlen, da die Trainingszeit hier direkt steuerbar ist. Werden schnelle Ergebnisse gewünscht und ist eine geringere Genauigkeit tolerabel so kann die Trainingszeit kurz gewählt werden. Sollen die Ergebnisse jedoch genau sein und eine höhere Laufzeit kann in Kauf genommen werden, dann kann eine längere Trainingszeit einberaumt werden.\\
Die Klassifikation mittels Support Vector Machine ist merkbar ungenauer als die drei hier zuerst genannten Verfahren und deshalb für die Trollerkennung eher zweite Wahl. Die Naive Bayes-Klassifikation ist als das mit Abstand ungenauste Verfahren am wenigsten für die Anwendung geeignet.